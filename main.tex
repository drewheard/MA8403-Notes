\documentclass{willowtreebook}
\usepackage{url}
\usepackage{tikz}
\usetikzlibrary{shapes}
\usetikzlibrary{decorations.pathreplacing}
\colorlet{lgray}{gray!25}
\colorlet{llgray}{gray!50}
\usepackage{tikz-cd}
\usepackage{cleveref}

\newcommand{\nc}{\newcommand}
\nc{\dmo}{\DeclareMathOperator}
\nc{\bbZ}{\mathbb{Z}}
\nc{\cat}[1]{\mathscr{#1}}
\nc{\Mid}{\,\big|\,}
\nc{\SET}[2]{\big\{\,#1\Mid#2\,\big\}}

\dmo{\End}{End}
\dmo{\Mod}{Mod}
\dmo{\Grp}{Grp}
\dmo{\Aut}{Aut}
\dmo{\Iso}{Iso}
\dmo{\sgn}{sgn}
\dmo{\Hom}{Hom}
\dmo{\id}{id}
\Crefname{example}{Example}{Examples}
\Title{MA8403  - Equivariant homotopy theory}
\Author{Drew Heard}
\BibliographyFile{ma8403} 		
	% The name of the .bib file, without file extension.
\begin{document}
\chapter{Preface}
This are the courses notes for MA8403 - Equivariant homotopy theory, held during the Autumn semester 2023 at NTNU. The notes are mainly based on two excellent sets of lectures notes, one by Guillou \cite{guillou}, and one by Blumberg \cite{blumberg}. The notes will be continually updated during the semester. 
\afterpreface
\chapter{Equivariance in algebra}
\subsection{Group actions in algebra}
We recall that if $X$ is an object in a category $\cat C$, then the set of endomorphisms $\End(X)$ forms a monoid (that is, a set equipped with an associative binary operation and an identity element) under composition. The set of automorphisms of $X$ (that is, those endomorphisms that are invertible) form a group. Moreover, we have
\[
\Aut(X) = \End(X) \cap \Iso(\cat C)
\]

Note that any group is a monoid, simply by forgetting the existence of inverses. 
\begin{definition}
    An action of a group $G$ on an object $X \in \cat C$ is a monoid homomorphism $a \colon G \to \End(X)$, or equivalently a group homomorphism $a \colon G \to \Aut(X)$ (note that a monoid homomorphism between groups is a group homomorphism). 
\end{definition}
\begin{remark}
    Unwinding the definition, this means that we have:
    \begin{enumerate}
        \item For each $g \in G$, there is a morphism $a(g) \colon G \to G$.
        \item $a$ preserves composition, i.e., $a(g \cdot h) = a(h) \cdot a(h)$. 
        \item $a$ preserves identifies, so $a(e) = \id_X$.
    \end{enumerate}
\end{remark}
\begin{example}
    Let $\cat C$ be the category of sets and functions. Then, $a \colon G \to \End(X) = \big\{\,f \colon X \to X\big\}$ correspond to a function $\overline{a} \colon G \times X \to X$. The conditions above mean that the diagrams
    % https://q.uiver.app/#q=WzAsNCxbMCwwLCJHIFxcdGltZXMgRyBcXHRpbWVzIFgiXSxbMCwxLCJHIFxcdGltZXMgWCJdLFsxLDAsIkcgXFx0aW1lcyBYIl0sWzEsMSwiWCJdLFswLDEsIlxcaWQgXFx0aW1lcyBcXG92ZXJsaW5le2F9IiwyXSxbMCwyLCJtIFxcdGltZXMgXFxpZCJdLFsyLDMsIlxcb3ZlcmxpbmV7YX0iXSxbMSwzLCJcXG92ZXJsaW5le2F9IiwyXV0=
\[\begin{tikzcd}[ampersand replacement=\&]
	{G \times G \times X} \& {G \times X} \\
	{G \times X} \& X
	\arrow["{\id \times \overline{a}}"', from=1-1, to=2-1]
	\arrow["{m \times \id}", from=1-1, to=1-2]
	\arrow["{\overline{a}}", from=1-2, to=2-2]
	\arrow["{\overline{a}}"', from=2-1, to=2-2]
\end{tikzcd}
\text{ and }
% https://q.uiver.app/#q=WzAsNCxbMCwwLCJcXHtcXGFzdFxcfSBcXHRpbWVzIFgiXSxbMSwwLCJHIFxcdGltZXMgWCJdLFswLDEsIlgiXSxbMSwxLCJYIl0sWzAsMSwiZSBcXHRpbWVzIFxcaWQiXSxbMCwyLCJcXGNvbmciLDJdLFsxLDMsIlxcb3ZlcmxpbmV7YX0iXSxbMiwzLCIiLDIseyJsZXZlbCI6Miwic3R5bGUiOnsiaGVhZCI6eyJuYW1lIjoibm9uZSJ9fX1dXQ==
\begin{tikzcd}[ampersand replacement=\&]
	{\{\ast\} \times X} \& {G \times X} \\
	X \& X
	\arrow["{e \times \id}", from=1-1, to=1-2]
	\arrow["\cong"', from=1-1, to=2-1]
	\arrow["{\overline{a}}", from=1-2, to=2-2]
	\arrow[Rightarrow, no head, from=2-1, to=2-2]
\end{tikzcd}\]
commute, where $m \colon G \times G \to G$ denotes the group multiplication. Equivalently, in symbols, we have
\[
\overline{a}(h,\overline{a}(h,x)) = \overline{a}(gh,x) 
\]
and
\[
\overline{a}(e,x) = x. 
\]
\end{example}
\begin{remark}
    Let $BG$ denote the category with one object $\ast$ and with $\Hom(\ast,\ast) = G$. Then an action of $G$ in the category $\cat C$ is the same as a functor $\rho \colon BG \to \cat C$. The object $X$ in the previous definition is the object $\rho(\ast) \in \cat C$. 
\end{remark}
\begin{remark}
    Let $\cat C = \Mod_R$ for a commutative ring $R$. An action of $G$ on $M \in \Mod_R$ is a monoid homomorphism
    \[
a \colon G \to \Hom_R(M,M).
    \]
    We recall that $\Hom_R(M,M)$ actually has the structure of an $R$-algebra
\end{remark}
\begin{definition}
    The ($R$-linear) group ring on $R$ is the $R$-algebra $R[G]$ whose:
    \begin{enumerate}[label=(\alph*)]
    \item underlying $R$-module is the free $R$-module with basis on the underlying set of $G$. 
    \item whose multiplication is given on basis elements by the group operation. 
    \end{enumerate}
\end{definition}
\begin{example}
    Let $R = \mathbb{Z}$ and $G = C_2 = \langle \sigma \rangle$. An element of $\bbZ[C_2]$ is of the form $a + b\sigma$ where $a,b \in \bbZ$. Multiplication is given by
    \[
(a_1 \cdot 1+b_1\sigma)\cdot(a_2\cdot 1+b_2\sigma) = (a_1a_2+b_1b_2)\cdot 1  + (a_1b_2+b_1a_2)\sigma. 
    \]
    This is the same thing as the polynomial ring $\bbZ[\sigma]/(\sigma^2-1)$. 
\end{example}
\begin{remark}
    Categorically, the group ring construction is left adjoint to the functor that takes an $R$-algebra to its group of units, i.e., there is an adjudication
    \[
R[-] \colon \Grp \leftrightarrows \Mod_R \colon (-)^{
\times}
    \]
\end{remark}
Returning to group actions, we have the following:
\begin{proposition}
    Let $R$ be a commutative ring, and $G$ a finite group. The following data on an $R$-module $M$ are equivalent:
    \begin{enumerate}
        \item A monoid homomorphism $G \to \End_R(M)$.
        \item A group homomorphism $G \to \Aut_R(M)$. 
        \item A homomorphism of $R$-algebras $R[G] \to \Hom_R(M,M)$. 
        \item An $R[G]$-module structure on $M$ whose underlying $R$-module structure is $M$. 
    \end{enumerate}
\end{proposition}
\begin{definition}
    A representation of $G$ over $R$ is an $R[G]$-module. 
\end{definition}
\begin{example}
    If $R = k$ is a field, then the underlying $R$-module is a $k$-vector space $V$. If $\dim_k(V) = n$, then $\Aut_k(V) = GL_n(k)$, and a $k$-representation is the same thing as a group homomorphism $G \to GL_n(k)$. 
\end{example}
\begin{definition}
    The $R[G]$-module $R[G]$ is known as the regular representation. More generally, if $X$ is a finite $G$-set, then the free $R$-module $R[X]$ inherits the structure of a $R[G]$-module (the case of $R[G]$ itself corresponds to the finite $G$-set $G$, considered as an $R[G]$-module over itself). Representations obtained this way are known as permutation representations. 
\end{definition}
\begin{example}
    Taking $X$ to be the trivial $G$-set, we obtain the (one-dimensional) trivial representation. This is simply the $R[G]$-module $R$, where $G$ acts trivially. 
\end{example}
\begin{definition}
    Let $G = C_2 = \langle \tau \rangle$, and suppose that $-1 \ne 1 \in R$. Then the sign representation of $G$ is the one-dimensional representation where $\tau$ acts as $-1$ (if $-1 = 1$ in $R$ this still makes sense, but is just the trivial representation). Note that this is an example of a representation that is not a permutation representation.
\end{definition}
\begin{example}
    Let $C_n = \langle \sigma \rangle$ be the cyclic group of order $n$. Let us calculate all complex 1-dimensional representations of $C_n$, i.e., homomorphisms $\rho \colon C_n \to \mathbb{C}$. Note that if we define $\rho(\sigma) =c$, then $\rho(\sigma^n) = c^n = 1$, so that $c$ must be an $n$-th root of unity. There are precisely $n$-of these (take $\zeta_n = e^{2\pi/n i}$), and so there are precisely $n$-representations. For example, when $G = C_4$, the four representations correspond to sending $\sigma$ to either $1,i,-1$ or $-i$. Note that we can also consider these as 2-dimensional \emph{real} representations. 
\end{example}
\begin{notation}
We let $\rho = \rho_G$ denote the regular representation of $G$, and the trivial $n$-dimensional representation by $\mathbf{n} = R^{\oplus n}$. 
\end{notation}
\begin{definition}
    A subrepresentation is a submodule. 
\end{definition}
\begin{example}
    The regular representation always has a one-dimensional trivial representation, generated by the sum $\sum_{g \in G}g$. 
\end{example}
\begin{definition}
    A representation $V$ is irreducible if the only subrepresentations of $V$ are $0$ and $V$. 
\end{definition}
\begin{theorem}[Maschke]Suppose that $k$ is a field of characteristic not dividing $|G|$. Then every
representation splits as a direct sum of irreducible representations.
\end{theorem}
\begin{proof}
    We prove the following: if $V \subseteq W$ is a subrepresentation, then there exists $U \subseteq W$ such that $U \oplus V \simeq W$. 

    To see this, let $\pi \colon W \to W$ be any $k$-linear projection of $W$ onto $V$. This map need not be $G$-equivariant, but we can make it so by `averaging'. That is, we define a new map $\phi \colon W \to W$ by 
    \[
\phi(\mathbf{w}) = \frac{1}{|G|}\sum_{g \in G}g \cdot \pi(g^{-1} \cdot \mathbf{w}). 
    \]
   Moreover the map is $G$-equivariant: for $h \in G$ we have
   \[
   \begin{split}
\phi(h \cdot \mathbf{w}) &= \frac{1}{|G|}\sum_{g \in G}g \cdot \pi(g^{-1} \cdot h \cdot \mathbf{w}) \\
& = \frac{1}{|G|}\sum_{u \in G}u \cdot h \cdot \pi(u^{-1} \cdot \mathbf{w}) \\
&= h \cdot \phi(\mathbf{w}),
\end{split}
   \]
   where $u = gh^{-1}$, and so $\phi$ is $k[G]$-linear. Furthermore, the map $\phi$ is the identity on $V$ By the splitting lemma, $W = V \oplus \ker(\phi)$. 
\end{proof}
\begin{remark}
We used the assumption on $k$ to ensure that we could divide by $|G|$. Without that assumption, the theorem is false. Indeed, let $G = C_2$, $R = \mathbb{F}_2$ and consider the representation defined by $\rho(\tau) =\begin{pmatrix}
1 & 1 \\
1 & 0 
\end{pmatrix} $. This is not irreducible, but does not split as a direct sum of indecomposable representations. 
\end{remark}
\begin{corollary}
Suppose that $k$ is a field of characteristic not dividing $|G|$.  If $V$ is an irreducible representation, then $V$ is isomorphism to a subgroup of $k[G]$ (slogan: all irreducibles are submodules of the regular representation). 
\end{corollary}
\begin{proof}
    Let $\mathbf{v} \in V$ be non-trivial. Then the homomorphism $\phi \colon k[G] \to V$ given by sending $1$ to $\mathbf{v}$ must be surjective, because $V$ is irreducible. Let $U = \ker(\phi)$, then apply Maschke's theorem. 
\end{proof}
\begin{example}
    Let $G = C_2 = \langle \tau \rangle$, and $k$ a field of characteristic not equal to 2. We have the trivial representation $\mathbf{1}$ and the sign representation $\mathbf{1}_{\sgn}$. Then, $\mathbf{1}$ is generated by the sum $1+\tau$, while $\mathbf{1}_{\sgn}$ is generated by $1-\tau$, and we deduce that 
    \[
\rho_{C_2} = \mathbf{1} \oplus \mathbf{1}_{\sgn}. 
    \]
\end{example}
\section{The representation ring}
Let $k$ be a field, and suppose that $V$ and $W$ are $G$-representations, then the $k$-linear tensor sum $V \oplus W$ can be given the structure of a $k[G]$-module, by taking the diagonal $G$-action. If we think of a representation in terms of a homomorphism $\rho \colon G \to GL_n(k)$, then this direct sum corresponds to the `block sum'
\[
G \to GL_n(k) \times GL_m(k) \to GL_{n+m}(k)
\]
Similarly, we can define a tensor product of representations using the `Kronecker tensor product' (or matrix direct product). Equivalently, this is the $k$-linear tensor product $V \otimes W$ with the $g$-action defined on simple tensors by
\[
g \cdot (\mathbf{v} \otimes\mathbf{w}) = g \cdot \mathbf{v} \otimes g \mathbf{w}. 
\]
We leave it for the reader to verify the following straightforward computations:
\begin{enumerate}[label=(\alph*)]
    \item $\mathbf{1} \otimes V \cong V \cong V \otimes \mathbf{1}$. 
    \item $\mathbf{n} \otimes V \cong V^{\oplus n} \cong V \otimes \mathbf{n}$. 
\end{enumerate}
\begin{example}[label=ex:tensor-product-sign]
    Let us compute the tensor product $\mathbf{1}_{\sgn} \otimes \mathbf{1}_{\sgn}$. The underlying vector space is simply $k \otimes k \cong k$, while $\tau$ acts as $\tau \cdot (1 \otimes 1) = (\tau \cdot 1) \otimes (\tau \cdot 1) = -1 \otimes -1 = 1 \otimes 1$. So the tensor product $\mathbf{1}_{\sgn} \otimes \mathbf{1}_{\sgn} = \mathbf{1}$. 
   
\end{example}
\begin{example}[label=ex:tensor-product-c3]
    Take $G = C_3$ and $k = \mathbb{R}$. We have a two-dimensional representation $\lambda_3$ corresponding to rotation by an angle of $2\pi/3$. What is $\lambda_3 \otimes \lambda_3$? This is a 4-dimensional representation, and by working out all irreducible representations must be either $\mathbf{4}, \mathbf{2} \oplus \lambda_3$ or $\lambda_3 \oplus \lambda_3$. If you know a little bit of character theory, you can see that it must be $\mathbf{2} \oplus \lambda_3$: we have
    \[
\chi_{\lambda_3 \otimes \lambda_3}(1) = 4, \quad \chi_{\lambda_3 \otimes \lambda_3}(\tau) = 1
    \]
        \[
\chi_{\mathbf{4}}(1) = 4, \quad \chi_{\mathbf{4}}(\tau) = 1
    \]
        \[
\chi_{\mathbf{2} \oplus \lambda_3}(1) = 4, \quad \chi_{\mathbf{2} \oplus \lambda_3}(\tau) = 1
    \]
            \[
\chi_{\lambda_3 \oplus \lambda_3}(1) = 4, \quad \chi_{\lambda_3  \oplus \lambda_3}(\tau) = -2
    \]
\end{example}
\begin{remark}
    By passing to isomorphism classes of representations, the set of finite dimensional representations has the structure of a semiring. Using the Grothendieck construction, we can produce a commutative ring. 
\end{remark}
\begin{definition}
    For a finite group $G$ the real representation ring $RO(G)$ is the Grothendieck group of the above semi-ring. Explicitly,
    \[
RO(G) \coloneqq \mathbb{Z}\left\{ \parbox{15em}{isomorphism classes of finite-dimensional real $G$-representations} \right\}/\langle [ V \oplus W] - [V] - [W] \rangle .
    \]
\end{definition}
\begin{remark}
    As an abelian group, $RO(G)$ is a direct sum of copies of $\bbZ$, with rank equal to the number of isomorphism classes of irreducible representations. 
\end{remark}
\begin{remark}
    We can make the same definition for other fields, for example when $k = \mathbb{C}$ we get the complex representation ring $R(G)$. 
\end{remark}
\begin{example}
    We have $RO(C_2) \cong \bbZ\{ \mathbf{1} \} \oplus \bbZ\{\mathbf{1}_{\sgn}\}$. The ring structure is determined by Example~\eqref{ex:tensor-product-sign}: we have $RO(C_2) \cong \bbZ[\sigma]/(\sigma^2-1)$. The same is true for the complex representation ring. Note that this is the same as $\bbZ[C_2]$. In fact, the complex representation ring of a finite abelian group is always (non-canonically) isomorphic to the group ring: it is the group ring of the character group. 
\end{example}
\begin{example}
    When $G = C_3$ we have that
    \[
RO(C_3) = \bbZ\{ \mathbf{1} \} \oplus \bbZ\{ \lambda_3 \}. 
    \]
    By Example~\eqref{ex:tensor-product-c3} we have $[\lambda_3]^2 = 2 + [\lambda_3]$ and we see that
    \[
RO(C_3) \cong \bbZ[\lambda]/(\lambda^2-\lambda-2). 
    \]
    On the other hand, the complex representation ring is given by \[
    R(C_3) \cong \bbZ[\zeta]/(\zeta^3-1).
    \]
    By tensoring a real representation with $\mathbb{C}$, there is a map
    \[
RO(C_3) \to R(C_3)
    \]
    given by $\lambda \mapsto \zeta + \zeta^2$. 
\end{example}

\par\bigskip\noindent

	% End the document without loading the bibliography
	% or the index, or the list of notation.
\end{document}